\section{Literatura}

\appendix

\section{Návod pro vkládání grafů a obrázků do Latexu}

Nejjednodušší způsob vytvoření obrázku je použít vektorový grafický
editor (např. xfig nebo jfig), ze kterého lze exportovat buď
\begin{itemize}
\item postscript formáty (ps nebo eps formát) nebo
\item latex formáty (v pořadí prostý latex, latex s macry epic, eepic, eepicemu). Uvedené pořadí odpovídá růstu
komplikovanosti obrázků který formát podporuje (prostá latex macra
umožnují pouze jednoduché, epic makra něco mezi, je třeba
vyzkoušet).

\end{itemize}
Následující příklady platí pro všechny případy.

Obrázek v postscriptu, vycentrovaný a na celou šířku stránky, s
popisem a číslem. Všimnete si, jak řídit velikost obrazku.
\begin{figure}[ht]
\epsfysize=3cm \centerline{\epsfbox{VasObrazek.ps}} \caption{Popis
vašeho obrazku} \label{labelvasehoobrazku}
\end{figure}

Obrázek pouze vložený mezi řádky textu, bez popisu a číslování.\\
\epsfxsize=1cm
\rule{0pt}{0pt}\hfill\epsfbox{VasObrazek.ps}\hfill\rule{0pt}{0pt}

Latexovské obrázky maji přípony *.latex, *.epic, *.eepic, a
*.eepicemu, respective.
\begin{figure}[ht]
\begin{center}
\input VasObrazek.latex
\end{center}
\caption{Popis vašeho obrázku} \label{l1}
\end{figure}
Vypuštením závorek {\tt figure} dostanete opět pouze rámeček v textu
bez čísla a popisu.

Takhle jednoduše můžete poskládat obrázky vedle sebe.
\begin{center}
\setlength{\unitlength}{0.1mm}\input VasObrazek.epic
\hglue 5mm
\setlength{\unitlength}{0.15mm}\input VasObrazek.eepic
\hglue 5mm
\setlength{\unitlength}{0.2mm}\input VasObrazek.eepicemu
\end{center}
Řídit velikost latexovskych obrázků lze příkazem
\begin{verbatim}
\setlength{\unitlength}{0.1mm}
\end{verbatim}
které mění měřítko rastru obrázku, Tyto příkazy je ale současně
nutné vyhodit ze souboru, který xfig vygeneroval.

Pro vytváření grafu lze použít program gnuplot, který umí generovat
postscriptovy soubor, ktery vložíte do Latexu výše uvedeným
způsobem.